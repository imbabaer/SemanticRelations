\documentclass[12pt,a4paper]{article}


\usepackage[margin=3cm]{geometry}
\usepackage{ngerman}
\usepackage[utf8]{inputenc}
\usepackage[onehalfspacing]{setspace}


\begin{document}
\tableofcontents
\newpage
\section{Einleitung}
	\subsection{Motivation}
	\subsection{Problemstellung}
	\subsection{Aufbau der Arbeit}
\newpage
\section{Begriffverzeichnis}
	\begin{tabular}{r|l}	
	\textbf{Begriff} & Erklärung\\
	\hline	
	\textbf{Ähnliche Worte} & Im Word2Vec-Modell mit der Methode $most\_similar()$ erhaltene Worte.\\
	\textbf{SVM} & Support Vector Machine\\
	\textbf{NBC} & Naive Bayes Classifier\\

\end{tabular}
\newpage
\section{Daten und Vorverarbeitung}
	\subsection{Datenbasis}
	\subsection{Externe Programme und Hilfsmittel}
	\subsection{Vorverarbeitung}
\newpage
\section{Word2Vec}
	\subsection{Parameter}
	\subsection{CBOW}
	\subsection{Skip-gram}
	\subsection{Negative sampling}
	\subsection{Hierarchical softmax}
	\subsection{Distanz zwischen Vektoren im Word2Vec Modell}
\newpage
\section{Wikipedia-Korpus}
	\subsection{Gesamtkorpus}
	\label{sec:Gesamtkorpus}
	\subsection{Teilkorpus}
	\label{sec:Teilkorpus}	
	\subsection{Testdaten}
	\subsection{Vergleich und Analyse}
\newpage
\section{Experimente}
In diesem Kapitel sollen die unterschiedlichen Korpora (Gesamtkorpus\footnote{vgl. \ref{sec:Gesamtkorpus}} und Techkorpus\footnote{vgl. \ref{sec:Teilkorpus}}) untersucht werden. Dies soll durch ausgewählte Fragestellungen realisiert werden.
\\Die Fragestellungen beziehen sich immer auf die Ergebnisse, die aus den Tastdaten\footnote{vgl. Testdaten unter \ref{sec:Testdaten}} erhaltenen ähnlichen Worten.
\\Jedes Experiment ist in drei Teile aufgeteilt Beschreibung, Durchführung und Interpretation/Ergebnis.
	\subsection{Synnonymsuche durch Rekursion}
		\subsubsection{Beschreibung}
		Es soll untersucht werden, ob man Synonyme zum Testwort erhält, wenn man die ähnlichen Worte dieses Testwortes erneut im Model mittels der Methode $most\_similar()$ sucht. 
		\subsubsection{Durchführung}
		.
		\subsubsection{Interpretation/Ergebnis}
		.
	\subsection{Konkretisierungen}
		\subsubsection{Beschreibung}
		.
		\subsubsection{Durchführung}
		.
		\subsubsection{Interpretation/Ergebnis}
		.
	\subsection{Verallgemeinerungen}
		\subsubsection{Beschreibung}
		.
		\subsubsection{Durchführung}
		.
		\subsubsection{Interpretation/Ergebnis}
		.
	\subsection{Unterschiedliche Beziehungen}
		\subsubsection{Beschreibung}
		.
		\subsubsection{Durchführung}
		.
		\subsubsection{Interpretation/Ergebnis}
		.
	\subsection{Mehrdeutigkeit}
		\subsubsection{Beschreibung}
		.
		\subsubsection{Durchführung}
		.
		\subsubsection{Interpretation/Ergebnis}
		.
		

\newpage
\section{Zusammenfassung}
\newpage
\section{Ausblick}
\newpage
\section{Fazit}
\newpage
\section{Anhang}
	\subsection{Testdaten}
	\label{sec:Testdaten}
	

	
	
\end{document}