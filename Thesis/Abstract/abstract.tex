\documentclass[11pt,a4paper]{article}


\usepackage[margin=3.5cm]{geometry}
\usepackage{ngerman}
\usepackage[utf8]{inputenc}
\usepackage[onehalfspacing]{setspace}

\begin{document}

\section*{Abstract}
Diese Bachelorthesis beschäftigt sich mit der Analyse von semantischen Beziehungen innerhalb mit Word2Vec gelernten Modellen.
\\Dazu sollen zum einen schon der vorhandene allgemeine Wikipedia-Korpus gelernt und analysiert werden, was als semantisch ähnlich erkannt wird. Zum anderen soll ein Korpus über eine spezielle Domäne erstellt und gelernt werden. Welche spezielle Domäne analysiert und verglichen werden soll, wird im Laufe der Bearbeitung festgelegt.
\\Diese beiden Korpora sollen sich dann gegenüber gestellt und analysiert werden, was jeweils als semantisch ähnlich erkannt wird. 
\\Ziel dieser Arbeit soll es sein, festzustellen ob ein allgemeiner Korpus oder ein spezieller Domänenkorpus genauere Resultate im Hinblick auf semantische Ähnlichkeiten erzielt. Anstatt eines allgemeinen Korpus zu verwenden, könnte es sich dann anbieten zwischen mehreren speziellen Korpora auszuwählen, je nachdem welche Domäne aktuell bearbeitet werden soll.



\end{document}