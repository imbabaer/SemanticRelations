\documentclass[12pt,a4paper]{article}


\usepackage[margin=3cm]{geometry}
\usepackage{ngerman}
\usepackage[utf8]{inputenc}
\usepackage[onehalfspacing]{setspace}


\begin{document}
\section{Experimente}
In diesem Kapitel sollen die unterschiedlichen Korpora (Gesamtkorpus\footnote{vgl. \ref{sec:Anhang}} und Techkorpus\footnote{vgl. \ref{sec:Anhang}}) untersucht werden. Dies soll durch ausgewählte Fragestellungen realisiert werden.
\\Die Fragestellungen beziehen sich immer auf die Ergebnisse, die aus den Tastdaten\footnote{vgl.Testdaten unter \ref{sec:Anhang}} erhaltenen ähnlichen Worten.
\\Jedes Experiment ist in drei Teile aufgeteilt Beschreibung, Durchführung und Interpretation/Ergebnis.
	\subsection{Synnonymsuche durch Rekursion}
		\subsubsection{Beschreibung}
		Es soll untersucht werden, ob man Synonyme zum Testwort erhält, wenn man die ähnlichen Worte dieses Testwortes erneut im Model mittels der Methode $most\_similar()$ sucht. 
		\subsubsection{Durchführung}
		.
		\subsubsection{Interpretation/Ergebnis}
		.
	\subsection{•}
		\subsubsection{Beschreibung}
		\subsubsection{Durchführung}
		\subsubsection{Interpretation/Ergebnis}
	\subsection{•}
		\subsubsection{Beschreibung}
		\subsubsection{Durchführung}
		\subsubsection{Interpretation/Ergebnis}
	\subsection{•}
		\subsubsection{Beschreibung}
		\subsubsection{Durchführung}
		\subsubsection{Interpretation/Ergebnis}
		
\section{Anhang}
\label{sec:Anhang}
	
\end{document}