\documentclass[12pt,a4paper]{report}
\usepackage[margin=3cm]{geometry}
\usepackage{ngerman}
\usepackage[utf8]{inputenc}
\usepackage[onehalfspacing]{setspace}
\begin{document}


\pagenumbering{roman}
\thispagestyle{empty}
\begin{verbatim}





\end{verbatim}
\begin{center}
\textbf{\huge{Semantische Beziehungen in Texten mit Word2Vec}}\\
\end{center}
\begin{verbatim}



\end{verbatim}
\begin{center}
\large B A C H E L O R A R B E I T

\begin{verbatim}
\end{verbatim}

\end{center}
\begin{center}
im Studiengang\\
\textsc{Medieninformatik (MI7)}\\
an der Hochschule der Medien in Stuttgart\\
vorgelegt von \textsc{Ruben Müller}\\
\begin{verbatim}
\end{verbatim}
im Juli 2015


\end{center}
\begin{verbatim}







\end{verbatim}

\begin{flushleft}
\begin{tabular}{lll}
\textbf{Erstprüfer:} & & \textsc{Prof. Dr-Ing. Johannes Maucher},\\ 
&&\small Hochschule der Medien, Stuttgart  \\
\textbf{Zweitprüfer:} & & \textsc{M.Sc. Andreas Stiegler},\\
&&\small Hochschule der Medien, Stuttgart\\
\end{tabular}
\end{flushleft}

\newpage
\chapter*{Erklärung}
Hiermit versichere ich, Ruben Müller, an Eides Statt, dass ich die vorliegende
Bachelorarbeit mit dem Titel: "Semantische Beziehungen in Texten mit Word2Vec" selbständig und ohne fremde Hilfe verfasst und keine anderen als die angegebenen
Hilfsmittel benutzt habe. Die Stellen der Arbeit, die dem Wortlaut oder dem Sinn nach anderen
Werken entnommen wurden, sind in jedem Fall unter Angabe der Quelle kenntlich gemacht. Die
Arbeit ist noch nicht veröffentlicht oder in anderer Form als Prüfungsleistung vorgelegt worden.\\
Ich habe die Bedeutung der eidesstattlichen Versicherung und die prüfungsrechtlichen Folgen (§ 23 Abs. 2 Bachelor-SPO (7 Semester) der HdM) sowie die strafrechtlichen Folgen (gem. § 156 StGB) einer unrichtigen oder
unvollständigen eidesstattlichen Versicherung zur Kenntnis genommen.\\
\vspace{1em}\\
Filderstadt, den XX. Juli 2015\\
\vspace{5em}\\
Ruben Müller


\newpage
\chapter*{Kurzfassung}
Diese Bachelorthesis beschäftigt sich mit der Analyse von semantischen Beziehungen innerhalb mit Word2Vec gelernten Modellen.
\\Dazu sollen zum einen schon der vorhandene allgemeine Wikipedia-Korpus gelernt und analysiert werden, was als semantisch ähnlich erkannt wird. Zum anderen soll ein Korpus über eine spezielle Domäne erstellt und gelernt werden. Welche spezielle Domäne analysiert und verglichen werden soll, wird im Laufe der Bearbeitung festgelegt.
\\Diese beiden Korpora sollen sich dann gegenüber gestellt und analysiert werden, was jeweils als semantisch ähnlich erkannt wird. 
\\Ziel dieser Arbeit soll es sein, festzustellen ob ein allgemeiner Korpus oder ein spezieller Domänenkorpus genauere Resultate im Hinblick auf semantische Ähnlichkeiten erzielt. Anstatt eines allgemeinen Korpus zu verwenden, könnte es sich dann anbieten zwischen mehreren speziellen Korpora auszuwählen, je nachdem welche Domäne aktuell bearbeitet werden soll.
\newpage
\chapter*{Abstract}
\newpage
\tableofcontents
\newpage
\chapter*{Begriffsverzeichnis}
	\begin{tabular}{r|l}	
	\textbf{Begriff} & Erklärung\\
	\hline	
	\textbf{Ähnliche Worte} & Im Word2Vec-Modell mit der Methode $most\_similar()$ erhaltene Worte.\\
	\textbf{SVM} & Support Vector Machine\\
	\textbf{NBC} & Naive Bayes Classifier\\

\end{tabular}
\newpage
\pagenumbering{arabic}
\chapter{Einleitung}
	\section{Motivation}
	
	\section{Problemstellung}
	\section{Aufbau der Arbeit}

\newpage
\chapter{Daten und Vorverarbeitung}
	\section{Datenbasis}
	Kompletter Wikikorpus: \\
	8392453 Artikel\\
	wordcount: 2919802692\\
	sentencecount: 242144317\\
	
	Techkorpus:\\
	wordcount: 9866096\\
	sentencecount: 3166065\\
	\section{Externe Programme und Hilfsmittel}
	Dieser Abschnitt enthält eine Auflistung mit kurzen Beschreibungen, der in dieser Arbeit verwendeten Hilfsmittel und externen Programme.\\
	\vspace{1em}\\	
	\textbf{gensim}\cite{}\footnote{https://radimrehurek.com/gensim/, abgerufen am 24.06.2015} ist eine Bibliothek für Python. Sie enthält unter anderem eine performanzoptimierte Implementierung von Word2Vec. 
	\vspace{1em}\\
	\section{Vorverarbeitung}
\newpage
\chapter{Word2Vec}
	\section{Parameter}
	\section{CBOW}
	\section{Skip-gram}
	\section{Negative sampling}
	\section{Hierarchical softmax}
	\section{Distanz zwischen Vektoren im Word2Vec Modell}
\newpage
\chapter{Wikipedia-Korpus}
	\section{Gesamtkorpus}
	\label{sec:Gesamtkorpus}
	\section{Teilkorpus}
	\label{sec:Teilkorpus}	
	\section{Testdaten}
	\subsection{Vergleich und Analyse}
\newpage
\chapter{Experimente}
In diesem Kapitel sollen die unterschiedlichen Korpora (Gesamtkorpus\footnote{vgl. \ref{sec:Gesamtkorpus}} und Techkorpus\footnote{vgl. \ref{sec:Teilkorpus}}) untersucht werden. Dies soll durch ausgewählte Fragestellungen realisiert werden.
\\Die Fragestellungen beziehen sich immer auf die Ergebnisse, die aus den Tastdaten\footnote{vgl. \ref{sec:Testdaten}} erhaltenen ähnlichen Worten.
\\Jedes Experiment ist in drei Teile aufgeteilt Beschreibung, Durchführung und Interpretation/Ergebnis.
	\section{Synnonymsuche durch Rekursion}
		\subsection{Beschreibung}
		Es soll untersucht werden, ob man Synonyme zum Testwort erhält, wenn man die ähnlichen Worte dieses Testwortes erneut im Model mittels der Methode $most\_similar()$ sucht. 
		\subsection{Durchführung}
		.
		\subsection{Interpretation/Ergebnis}
		.
	\newpage
	\section{Konkretisierungen}
		\subsection{Beschreibung}
		.
		\subsection{Durchführung}
		.
		\subsection{Interpretation/Ergebnis}
		.
	\newpage
	\section{Verallgemeinerungen}
		\subsection{Beschreibung}
		.
		\subsection{Durchführung}
		.
		\subsection{Interpretation/Ergebnis}
		.
	\newpage
	\section{Unterschiedliche Beziehungen}
		\subsection{Beschreibung}
		.
		\subsection{Durchführung}
		.
		\subsection{Interpretation/Ergebnis}
		.
	\newpage
	\section{Mehrdeutigkeit}
		\subsection{Beschreibung}
		.
		\subsection{Durchführung}
		.
		\subsection{Interpretation/Ergebnis}
		.
		

\newpage
\chapter{Fazit und Ausblick}
\section{Fazit}

\newpage
\section{Ausblick}
\newpage
\chapter*{Quellenverzeichnis}
\bibliographystyle{plain}
\bibliography{bibtex_gensim}
\nocite{rehurek_lrec}
papers und so
\listoftables
\listoffigures 




\chapter{Anhang}
	\section{Testdaten}
	\label{sec:Testdaten}
\begin{table}[h]
\caption{Testdaten Teil 1}
\begin{tabular}{l|l|l|l}\\
3d & 3ds & 3g & 4chan\\
4g & acer & acta & activision\\
adobe & amazon & android & anonymous\\
aol & apple & app & augmented\\
arcade & architecture & arpanet & asus\\
auto & automobile & battlefield & bing\\
biometrics & bitcoin & bittorrent & blackberry\\
blizzard & blogging & blog & bluray\\
broadband & browser & casual & chatroulette\\
chrome & chromebook & cispa & computing\\
console & cookies & craigslist & crowdfunding\\
crowdsourcing & cryptocurrency & cybercrime & cyberwar\\
darknet & data & dell & diablo\\
doodle & dotcom & drone & dropbox\\
e3 & ebay & email & emoji\\
encryption & energy & engine & engineering\\
ereader & events & facebook & fat\\
filesharing & firefox & flickr & foursquare\\
gadget & game & gameplay & gamergate\\

\end{tabular}
\end{table}
\newpage
\begin{table}[h]
\caption{Testdaten Teil 2}
\begin{tabular}{l|l|l|l}\\
games & gaming & ghz & gmail\\
google & googlemail & gps & groupon\\
gta & hacking & halo & handheld\\
hardware & hashtag & hd & heartbleed\\
htc & html5 & i & ibm\\
icloud & ie & imac & indie\\
instagram & intel & internet & ios\\
ipad & iphone & ipod & isp\\
itunes & keyboard & kickstarter & kindle\\
kinect & laptop & lenovo & lg\\
limewire & link & linkedin & linux\\
live & machinima & macintosh & macworld\\
malware & mario & megaupload & microsoft\\
minecraft & mmorpg & mobile & monitor\\
motoring & mouse & mozilla & myspace\\
nes & net & netbook & nfs\\
nintendo & nokia & oracle & ouya\\
p2p & paypal & pc & phablet\\
phishing & photography & photoshop & pi\\
pinterest & piracy & pirate & platform\\
playback & playstation & pokemon & power\\
processor & programming & ps & ps2\\
ps3 & ps4 & psp & python\\
raider & ram & raspberry & rayman\\
recommendation & reddit & retro & robot\\
rpg & rts & safari & samsung\\
search & security & seo & skype\\
smartphone & smartphones & smartwatch & smartwatches\\
software & sonic & sony & sopa\\
spam & spotify & steam & stream\\
starcraft & stuxnet & sun & surface\\
symbian & tablet & technology & technophile\\
ted & telecom & television & tetris\\
titanfall & tomb & trojan & tumblr\\
twitch & twitter & viber & vine\\
virus & warcraft & web & whatsapp\\
wheel & wifi & wii & wikipedia\\
windows & windows7 & wireless & worms\\
wow & xbox & xp & y2k\\
yahoo & youtube & zelda & zynga\\

\end{tabular}
\end{table}
	
	
\end{document}
